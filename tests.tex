% \documentclass{article}
% \usepackage{tikz}
% \usetikzlibrary{decorations.text,calc,arrows.meta}

% \documentclass{standalone}
% \usepackage{tikz}
% \begin{document}
% \begin{tikzpicture}
% \def\angle{60}%
% \pgfmathsetlengthmacro{\xoff}{2cm*cos(\angle)}%
% \pgfmathsetlengthmacro{\yoff}{1cm*sin(\angle)}%
% \draw (\xoff,\yoff) circle[x radius=8cm, y radius=4cm] ++(3*\xoff,3*\yoff) node{Non-Hermitian};

% \draw circle[x radius=5cm, y radius=2.5cm] ++(1.5*\xoff,1.5*\yoff) node{Hermitian};
% \end{tikzpicture}


% \documentclass{beamer}
% \usepackage{xmpmulti}

% \begin{document}

% \begin{frame}{Preparation}
% \begin{itemize}
% \item Make sure your images are named as \texttt{<filename>-<x>.<format>} where \texttt{x} is the sequence number.
% \item By default, \texttt{x} starts from 0, but you can specify a custom starting number.
% \item Example: \texttt{fig-0.png}, \texttt{fig-1.png}, \texttt{fig-2.png},\ldots
% \item The following example uses 8 images named as \texttt{Draw-a-Cat-Face-Step-1.jpg}, \texttt{Draw-a-Cat-Face-Step-2.jpg}\ldots
% \end{itemize}
% \end{frame}

% \begin{frame}
% \multiinclude[format=jpg,start=1,graphics={width=\textwidth}]{Draw-a-Cat-Face-Step}
% \end{frame}

\documentclass{article}
\usepackage[export]{adjustbox}

\begin{document}

\begin{figure}

\adjustimage{width=.6\textwidth,center,caption={A particle travels from A to B in time t. Can we make this trip nearly instantaneous?},figure}{\adjustimage{left}{\multiinclude[format=png,start=1,graphics={width=0.5\textwidth}]{optim-gif/optimisation}}


\end{figure}
\end{document}

